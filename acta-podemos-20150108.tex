% Created 2015-01-09 Fri 14:24
\documentclass[11pt]{article}
\usepackage[utf8]{inputenc}
\usepackage[T1]{fontenc}
\usepackage{fixltx2e}
\usepackage{graphicx}
\usepackage{longtable}
\usepackage{float}
\usepackage{wrapfig}
\usepackage{rotating}
\usepackage[normalem]{ulem}
\usepackage{amsmath}
\usepackage{textcomp}
\usepackage{marvosym}
\usepackage{wasysym}
\usepackage{amssymb}
\usepackage{hyperref}
\tolerance=1000
\usepackage[spanish]{babel}
\author{Podemos Villaciosa de Odón}
\date{\today}
\title{Acta Asamblea}
\hypersetup{
  pdfkeywords={},
  pdfsubject={},
  pdfcreator={Emacs 24.3.1 (Org mode 8.2.10)}}
\begin{document}

\maketitle
\tableofcontents

\section{Información}
\label{sec-1}
El jueves día 8 de Enero de 2015 tuvo lugar una asamblea de Podemos Villaviciosa de Odón, en el local cedido por la Asociación de Vecinos, a la que asistieron 18 personas, una de ellas por videoconferencia.
Comenzó a las 20:00 y terminó en torno a las 22:45 horas.

El orden del día fue el habitual:
\begin{itemize}
\item Recuento de la recaudación de las donaciones internas.
\item Estado de la preparación del próximo ejemplar del periódico.
\item Organización de las mesas informativas para el 10 y 11 de Enero de 2015.
\item Información sobre los trámites para la cesión de espacios públicos de cara al acto de constitución.
\item Ruegos y preguntas.
\end{itemize}

\section{Donaciones internas}
\label{sec-2}

La cuantía de las donaciones es de 515.05 euros, de las que hay que descontar 306.15, con lo que el resultado neto queda en 208.9 euros.

\section{Periódico}
\label{sec-3}

Javier del Castillo presenta el estado actual de la próxima edición del periódico, indicando que falta un texto de alrededor de 250 caracteres para la portada, y aportes más breves para rellenar varios huecos.

Se espera poder sufragar el coste de la impresión del siguiente ejemplar con la previsible recaudación de donativos en las mesas informativas.

Respecto a otras fuentes de financiación, Javier del Castillo dice que ha contactado con Autoescuela 101 de cara a incluirles como anunciantes.

Ana sugiere incluir algún artículo sobre el reciente atentado en París. En este sentido, informa de la receptividad favorable de villaviciosadigital.es para publicar contenido de Podemos en sus espacios online. También estarían dispuestos a pasarse y realizar fotos.

\section{Mesas informativas}
\label{sec-4}

Sergio Sánchez espera recibir el día siguiente (9 de Enero) tanto las chapas como las camisetas.
Las chapas se han encargado a chapea.com. El pedido de las camisetas asciende a 50 unidades: 4 de talla XL, 7 u 8 talla S, y el resto en tallas M o L.
Cada chapa tiene un coste de 0.18 euros, mientras que las camisetas cuestan 5.10 cada una, con IVA incluído en ambos casos.

A continuación se discute cómo organizarse de cara a las mesas informativas, tanto en el Villacenter como en el Centro Comercial El Bosque.
Se hace recuento de voluntarios para cada una de ellas, resultando en 8 personas disponibles para el sábado 10 de Enero, y 10 para el domingo 11 de Enero.

Tomás indica que es conveniente invitar a simpatizantes de otros Círculos, pero que en todo caso lo harían a título personal.

En Villacenter, para el sábado 10, se presentan voluntarios Viarce, Manuel, Tomás y César. El domingo 11 asistirán Viarce, Marina, Javier del Castillo y Tomás.
Para la mesa informativa de El Bosque, el sábado 10 a\-cu\-di\-rán Sergio, Ana Belén, Gumer, Ana y Abdul Karoni. Para la convocatoria del domingo 11 se prevé la asistencia de César, Javier del Castillo, Gumer y Ana Belén. Tanto Ana como Abdul intentarán asistir pero no pueden asegurarlo.

El horario acordado es de 10:00h en adelante, aunque se prevé que no habrá actividad a partir de las 15:00h. Los preparativos requieren cierta antelación, con lo que darán comienzo a las 9:15h.

Sergio recuerda que hay que llevar la pertinente autorización, y que el espacio público ocupado se ciña a las aceras.

Ana retomará la búsqueda de globos y dulces morados en comercios en el municipio, después de los intentos infructuosos de Javier del Castillo.

El debate se centró a continuación en torno a los carteles a colocar en las mesas.

\subsubsection{Cartel para la donación con regalo de una chapa}
\label{sec-4-0-1}

En el texto propuesto por Tomás se indica explícitamente el coste por chapa de 0.18 euros. Dado el precio, el texto se da por aceptado aunque sin votación.

\subsubsection{Cartel para la donación con regalo de una camiseta}
\label{sec-4-0-2}

Se vota mayoritariamente (12 votos) que el coste de la camiseta sea explícito. Tomás indica que en actos de otros círculos se sugiere una cantidad fija de 9 euros. César recuerda que no podemos llevar a engaño al simpatizante y recalcar que siempre se trata de un donativo.

Sobre el texto original planteado por Tomás, se hacen tres propuestas. La primera, consistente en reemplazar todo el texto por ``Ayúdanos a seguir adelante'', finalmente no recibe ningún voto. La segunda, propuesta por Manuel, que utiliza el texto ``Súbenos la oferta'', recibe un apoyo de 3 votos.
Finalmente la mayoría (14 votos) prefiere la opción de utilizar la expresión ``Con un poco más, nos ayudas a seguir adelante''. Se recoge 1 abstención.

\subsubsection{Cartel para la invitación a la participación}
\label{sec-4-0-3}

Ante la pregunta de por qué el texto propuesto por Tomás no incluye el lugar de celebración, y sí la hora y el día en el que tienen lugar las asambleas, éste argumenta que sirve al propósito de incitar al acercamiento y generar así oportunidades para el contacto.
El enfoque tiene aparente respaldo y nadie promueve una votación al respecto, por lo que se mantiene el texto original.

Tomás quiere informar de que, por un error inintencionado, en el logo de Podemos particularizado para el municipio aparece una ornamentación sobre el castillo que se votó eliminar.

\subsubsection{Nota de prensa}
\label{sec-4-0-4}

César toma la palabra y emplaza a redactar y enviar una nota de prensa para que los medios locales se hagan eco de las mesas informativas. Para la redación, Ana R. y Aída se ofrecen voluntarias. César sugiere continuar con la reunión y en paralelo redactar variantes de la nota de prensa, para realizar una votación a su término.

Ana propone adjuntar, a la nota de prensa, el panfleto que se va a u\-sar para recoger iniciativas. Tomás por su parte sugiere enviársela no sólo a villaviciosadigital.es y villaviciosadeodon.es, sino también a la propia Asociación de Vecinos (vecinosvilla.es).

\section{Solicitud de espacios públicos}
\label{sec-5}

Gumer toma la palabra y explica que el Centro de Mayores nos lo están denegando argumentando que debemos acudir a la Consejería de Atención Ciudadana, la cual ya nos rechazó la solicitud al no tener entidad jurídica. El único espacio público al que se nos daría acceso sería el Salón Cívico, que no está disponible. El Centro de Mayores, por otro lado, está casi siempre disponible durante los fines de semana.

Tomás indica que es posible que tengamos que registrarnos como Delegación de Podemos del municipio, en el Ministerio del Interior.

Manuel toma la palabra para preguntar por la carta de presentación que se envió al alcalde, y si incluyó la solicitud de espacios públicos. Ana Belén leyó la carta en voz alta, y Tomás aseguró que no se incluyó ninguna solicitud a ese respecto.

A continuación Sergio pregunta sobre si hay avances en los trámites de constitución de la asociación, dado que la gestión se inició en noviembre. Abdul propone pedir a instancias de Podemos del ámbito estatal que presione para desatascar estas trabas burocráticas.

César opina que es inasumible mantener la fecha del 18 de Enero para el acto de constitución, dada la cercanía en el tiempo. Propone asimismo medidas de presión adicionales, como un acto presencial en el ayuntamiento.
Tomás se muestra de acuerdo y propone también convocar una ``rueda de masas''. Sergio se manifiesta en ese sentido, y cree que no debemos centrarnos únicamente en las vías burocráticas, sino que deben ir acompañadas de otras medidas.
Gumer también opina que los trámites burocráticos no deben frenar otras posibles acciones. Otros asistentes (Javier del Castillo, Ana, Aída) se muestran a favor. Se recogen ideas para denunciar públicamente el uso antidemocrático del aparato municipal: reivindicaciones en el periódico, mesas informativas periódicas, etc.

Ana R. prefiere que el acto de constitución se realice en la plaza del ayuntamiento antes que en un local cerrado. 

Javier invita a revisar la ley de procedimiento administrativo, por si se dan las premisas del ``silencio positivo''. Tomás argumenta que para eso es necesario que transcurran 3 meses.

Tomás indica que los actos en la calle hay que notificarlos, y requieren medios y tiempo.

Gumer recuerda que en su momento se notificó a la concejala Lourdes Menéndez que el acto tendría lugar el día 18 de Enero, independientemente de tener concedido un espacio público.
Ricardo recuerda que en esas fechas (20 de Enero) se celebrará San Sebastián 2015, y podría ser contraproducente realizar el acto en fechas próximas.

En este momento muchos asistentes se van de la reunión, aunque ésta continúa.

\section{Redacción final de la nota de prensa}
\label{sec-6}

Se recogen tres propuestas para la nota de prensa. Finalmente sólo se vota la propuesta de César, que, al conseguir 5 votos, recibe el apoyo mayoritario dado el número de asistentes en ese momento.
El texto final de la nota de prensa dice así:

``¿Qué le falta a Villa? ¿Qué le sobra? Cuéntanoslo en las mesas que vamos a montar los días 10 y 11 de enero en el C.C. de El Bosque y en C.C. Villacenter a partir de las 10:00 hrs., y entérate de qué proponen los vecinos y vecinas de Podemos Villa''.

\section{Ruegos y preguntas}
\label{sec-7}

Aída toma la palabra y manifiesta su preocupación y malestar con el comportamiento de Gerardo al sembrar públicamente la sospecha en cuanto a los votos recibidos por Aída en las recientes votaciones al Consejo Ciudadano y Secretaría General.
Considera que son acusaciones muy graves que son objeto de denuncia frente a la Comisión de Garantías.

Otros asistentes opinan que es un conflicto de índole privada.

Ricardo argumenta que, sea o no de índole privada, ha tenido lugar en un espacio colectivo, y que no entiende el comentario de Gerardo sobre el recuento de votos.
Tomás indica que, sin posicionarse, él interpreta que el mensaje de Gerardo es una reflexión a raiz de deducir que 8 personas que votaron a Aída no votaron a ningún otro candidato, lo que supone el 20\% del total. Y que pueden ser votos de personas cercanas a Aída.
También, que la Comisión de Garantías no tiene competencia en este asunto dado que Gerardo ya no participa en el círculo.

Aída opina que es algo que no debe dejarse pasar para evitar que vuelva a suceder, y que el tono de los comentarios de Gerardo no es adecuado.

Gumer por su parte quiere dejar claro que no toma partido, pero no coincide con Aída en sus interpretaciones de los comentarios de Gerardo.
Jose, por otro lado, matiza que él sí coincide con las interpretaciones de Aída, y también cree que las sospechas de Gerardo son inaceptables.

\section{Próxima convocatoria}
\label{sec-8}

Si bien no se llegó a convocar la próxima convocatoria, se asume que será el jueves 15 de Enero de 2015, con el orden del día previsto:

\begin{itemize}
\item Recuento de las donaciones internas.
\item Análisis del acto de las mesas informativas.
\item Estado del periódico.
\item Solicitud de espacios públicos y otras acciones.
\item Ruegos y preguntas.
\end{itemize}
% Emacs 24.3.1 (Org mode 8.2.10)
\end{document}
